\documentclass[xcolor=x11names, xcolor=table]{beamer}
\usepackage[utf8]{inputenc}
\usepackage[T1]{fontenc}
\usepackage[ngerman]{babel}
\usepackage{minted}
\usepackage{graphicx}
%\usetheme{ufcd}
\definecolor{bgg}{rgb}{0.95,0.95,0.95}
\colorlet{bggg}{gray!30}
\setbeamercolor{block body}{bg=bgg,fg=black}
\setbeamercolor{block title}{bg=bggg,fg=black}
% the Madrid theme as custom theme:
%\usecolortheme{whale} % outer color
%\usecolortheme{orchid} % inner color
%\useinnertheme[shadow]{rounded}
%\useoutertheme{infolines}
%\usefonttheme{default}


%==============================================================================%
\begin{document}

\author{Tim Schulte}
\institute[Uni Freiburg]{
    Albert-Ludwigs-Universität Freiburg\\
    Grundlagen der Künstlichen Intelligenz
}
\title{The \LaTeX\ Beamer Class}
\date{\today}


% ------------------------------------------------------------------------------
\begin{frame}[plain]
    \titlepage
\end{frame}


\section{Introduction}
\subsection{Introduction}
% ------------------------------------------------------------------------------
\begin{frame}{Introduction}
    \begin{itemize}
            \item
    The \LaTeX\ beamer class enables the creation of presentations, handouts and
    articles from the same source
        \item Slides of size 128x96 mm
        \item \emph{Overlay} and \emph{theme} support
        \item Regular \LaTeX\ commands work
    \end{itemize}
\end{frame}


\section{Basics}
\subsection{Basics}
% ------------------------------------------------------------------------------
\begin{frame}[fragile]{The first slide}
    \begin{block}{slide.tex}
        \inputminted{latex}{tex/first-slide.tex}
    \end{block}
\end{frame}


% ------------------------------------------------------------------------------
\begin{frame}[fragile]{Beamer class options}
    \begin{minted}[autogobble]{latex}
        \documentclass[options]{beamer}
    \end{minted}

    \begin{block}{Options}
    \begin{itemize}
        \item \texttt{8pt, 9pt, 10pt, 11pt, 12pt, 14pt, 17pt, 20pt}
        \item \texttt{draft} -- no graphics, footlines, \dots
        \item \texttt{handout} -- no overlays
    \end{itemize}
    \end{block}

    The \texttt{beamer} class automatically loads packages \texttt{xcolor,
    amsmath, amsthm, calc, geometry, hyperref, extsizes}, and others
\end{frame}


% ------------------------------------------------------------------------------
\begin{frame}[fragile]{Frames}
    \begin{minted}[autogobble]{latex}
        \begin{frame}[options]{Title}{Subtitle}
        ...
        \end{frame}
    \end{minted}

    \begin{block}{Options}
    \vspace{.5em}
    \begin{tabular}{ll}
        \texttt{plain} & no headlines, footlines, sidebars\\
        \texttt{squeeze} & squeeze all vertical space\\
        \texttt{shrink=0..100} & shrink everything by $n$ percent\\
        \texttt{b, c, t} & vertically align at bottom, center, top\\
        \texttt{fragile} & \alert{required} when using macros (like
        \texttt{verbatim})\\
        \texttt{allowframebreaks} & auto-create new frames when necessary\\
        %if there is too much content to be displayed on one\\
        \texttt{label=...} & label frame for reuse with
        \mintinline{latex}{\againframe}
    \end{tabular}

    \vspace{.5em}
    Subtitle and title are optional
    \end{block}
\end{frame}


% ------------------------------------------------------------------------------
\begin{frame}{Titlepage}
    \begin{block}{slide-titlepage.tex}
        \inputminted{latex}{tex/slide-titlepage.tex}
    \end{block}
\end{frame}


% ------------------------------------------------------------------------------
\begin{frame}[fragile]{Environments}
\begin{itemize}
    \item Create bullet points with \texttt{itemize} and \texttt{enumerate}
    environments
    \item Include graphics with \texttt{figure} environment
    \item Include tables with \texttt{table} environment
    \item Include unprocessed text with \texttt{verbatim} environment
    \item Include math with \verb+\(...\)+ or \verb+\[...\]+ environments
\end{itemize}
\end{frame}


% ------------------------------------------------------------------------------
\begin{frame}[fragile]{Blocks}
    Emphasize important information using \texttt{block}, \texttt{alertblock},
    or \texttt{exampleblock} environments.


    \begin{columns}
    \column{.45\textwidth}
    \begin{minted}[autogobble]{latex}
        \begin{block}{title}
        ...
        \end{block}
    \end{minted}
    \column{.45\textwidth}
    \begin{block}{This is a Block}
    This is important information
    \end{block}
    \end{columns}

    \begin{columns}
    \column{.45\textwidth}
    \begin{minted}[autogobble]{latex}
        \begin{alertblock}{title}
        ...
        \end{alertblock}
    \end{minted}
    \column{.45\textwidth}
    \begin{alertblock}{This is an Alert block}
    This is an important alert
    \end{alertblock}
    \end{columns}

    %\begin{columns}
    %\column{.45\textwidth}
    %\begin{minted}[autogobble]{latex}
    %    \begin{exampleblock}{title}
    %    ...
    %    \end{exampleblock}
    %\end{minted}
    %\column{.45\textwidth}
    %\begin{exampleblock}{This is an Example block}
    %This is an example
    %\end{exampleblock}
    %\end{columns}

    \vspace{1em}
    Appearance of different block types is defined by style templates.
    Title is mandatory. Use \mintinline{latex}{\begin{block}{}} for empty title.
\end{frame}


% ------------------------------------------------------------------------------
\begin{frame}<0>{beamercolorbox, beamerboxesrounded}
\end{frame}


% ------------------------------------------------------------------------------
\begin{frame}<0>{scalebox, resizebox}
\end{frame}


% ------------------------------------------------------------------------------
\begin{frame}[fragile, c]{Columns}
%\scalebox{.8}{
\begin{columns}[c]
\column{.55\linewidth}
\begin{minted}[autogobble]{latex}
    \begin{columns}[c] %  t,b,h
        \column{.55\textwidth}
        ...
        \column{.35\textwidth}
        \includegraphics{gopher.png}
    \end{columns}
\end{minted}
\column{.35\linewidth}
    \includegraphics[width=.8\linewidth]{img/gopher.png}
\end{columns}
%}
\end{frame}


% ------------------------------------------------------------------------------
\begin{frame}[fragile]{Animations}
%\setbeamercovered{transparent}
Use the \mintinline{latex}{\pause} command to make items appear on a slide

\vspace{1em}
\begin{columns}
\column{.45\textwidth}
\begin{minted}[autogobble]{latex}
    We start our discussion
    with some concepts.
    \pause
    The first concept we
    introduce originates
    with Erd\H os.
\end{minted}
\column{.45\textwidth}
    \begin{overlayarea}{.9\textwidth}{5em}
    We start our discussion
    with some concepts.
    \only<2>
    {
    The first concept we
    introduce originates
    with Erd\H os.
    }
    \end{overlayarea}
\end{columns}

\vspace{2em}
Use \mintinline{latex}{\setbeamercovered{transparent}} or
\mintinline{latex}{\setbeamercovered{dynamic}} to show covered parts

\end{frame}


%TODO: handout mode
% ------------------------------------------------------------------------------
\begin{frame}[fragile]{Animations}
    \begin{columns}
    \column{.45\textwidth}
        \begin{minted}[autogobble]{latex}
            \begin{itemize}
                \item<1-> first
                \item<2-> second
                \item<3-> third
            \end{itemize}
        \end{minted}
    \column{.45\textwidth}
        \begin{minted}[autogobble]{latex}
            \begin{itemize}[<+->]
                \item first
                \item second
                \item third
            \end{itemize}
        \end{minted}
    \end{columns}

    \vspace{2em}
    Explicitly set up individual items to appear on a particular slide within
    the frame using \mintinline{latex}{<x-y>} syntax (left), or use shorthand
    (right) to auto-increment.
\end{frame}


% ------------------------------------------------------------------------------
\begin{frame}[fragile]{Animations}
    Most environments allow overlay specification
    \begin{minted}[autogobble]{latex}
        \begin{itemize}<1->
            \item ...
        \end{itemize}
    \end{minted}

    \begin{minted}[autogobble]{latex}
        \begin{block}<2>
            % only visible from second layer onwards
        \end{block}
    \end{minted}

    Many commands allow them, too

    \begin{minted}[autogobble]{latex}
        \textbf<3-4>{bold}
        \includegraphics<1,2>[scale=0.3]{gopher}
        ...
    \end{minted}
\end{frame}


% ------------------------------------------------------------------------------
\begin{frame}[fragile]{Animation}
Other useful commands to create animation/overlays

\begin{minted}[autogobble]{latex}
    \only<x-y>{ only visible on slide x to y }
\end{minted}
\begin{minted}[autogobble]{latex}
    \alt<x-y>{ visible x to y }{ visible else }
\end{minted}


The \texttt{overlayarea} environment prevents \emph{wobbling} slides when using
\mintinline{latex}{\only, \alt,} \dots

\begin{minted}[autogobble]{latex}
    \begin{overlayarea}{.9\textwidth}{5em}
        ...
    \end{overlayarea}
\end{minted}
\end{frame}


% ------------------------------------------------------------------------------
\begin{frame}[fragile]{Multimedia}
    Embedding videos is possible, using the \texttt{multimedia} package

    \vspace{1em}
    \begin{minted}[autogobble]{latex}
        \usepackage{multimedia}
        ...
        \movie[width=6cm, height=6cm, start=5s, duration=6s]
        {\includegraphics{image}}{movie.avi}
    \end{minted}

    \vspace{1em}
    But it might not work (depending on the PDF viewer)
\end{frame}


\section{Structure}
\subsection{Structure}
% ------------------------------------------------------------------------------
\begin{frame}[fragile]{Structuring}
    %As in all \LaTeX\ files,
    It is possible to structure the document using
    \begin{itemize}
        \item %
        \begin{minted}[autogobble]{latex}
        \section[Section]{My section}
        \end{minted}

        \item %
        \begin{minted}[autogobble]{latex}
        \subsection[Subsection]{My subsection}
        \end{minted}

        \item %
        \begin{minted}[autogobble]{latex}
        \subsubsection[Subsubsection]{My subsubsection}}
        \end{minted}
    \end{itemize}
    before and between frames

    \vspace{.5em}
    These commands do not generate text on the slides, but appear in the
    \mintinline{latex}{\tableofcontents}

    \vspace{.5em}
    In some themes, they appear in the sidebar or headline
\end{frame}


% ------------------------------------------------------------------------------
\begin{frame}[fragile]{Appendix}
\begin{minted}[autogobble]{latex}
    \section{...}
    \begin{frame} ... \end{frame}
    \appendix
    \begin{frame} ... \end{frame}
\end{minted}
Everything after \mintinline{latex}{\appendix} does not appear in the ToC

\vspace{.5em}
However, it affects the page counter
\end{frame}


\begin{frame}[fragile]{Hyperlinks}
First, add a target

\begin{minted}[autogobble]{latex}
\hypertarget{foo}{} or \hypertarget<2>{bar}{}
\end{minted}

Then, link it somewhere else
\begin{minted}[autogobble]{latex}
\hyperlink{foo}{\beamergotobutton{go to foo}}
\end{minted}

\begin{block}{Button types}
\mintinline{latex}{\beamergotobutton{foo}}\\
\mintinline{latex}{\beamerskipbutton{bar}}\\
\mintinline{latex}{\beamerreturnbutton{baz}}
\end{block}
\end{frame}


\section{Style}
\subsection{Style}
% ------------------------------------------------------------------------------
\begin{frame}[fragile]{Built-in themes}
    Beamer provides built-in themes % and color-themes% mostly named after cities

    \begin{minted}[autogobble]{latex}
        \usetheme{theme}
    \end{minted}

    \begin{block}{Themes}
    \begin{table}
    \begin{tabular}{llll}
        \texttt{default}  & \texttt{Copenhagen} & \texttt{Ilmenau}     & \texttt{Marburg}\\
        \texttt{Antibes}  & \texttt{Darmstadt}  & \texttt{JuanLesPins} & \texttt{Montpellier}\\
        \texttt{Bergen}   & \texttt{Dresden}    & \texttt{Luebeck}     & \texttt{PaloAlto}\\
        \texttt{Berkeley} & \texttt{Frankfurt}  & \texttt{Madrid}      & \dots\\
    \end{tabular}
    \end{table}
    \end{block}

\end{frame}


% ------------------------------------------------------------------------------
\begin{frame}[fragile]{The DIY approach}
You can create a custom theme by specifying
\begin{enumerate}
    \item a \texttt{colorscheme},
    \item an \texttt{innertheme},
    \item an \texttt{outertheme} and
    \item a \texttt{fonttheme} {\color{darkgray} (not covered here)}
\end{enumerate}
separately
\end{frame}


% ------------------------------------------------------------------------------
\begin{frame}[fragile]{Color theme}
    Choose a predefined \texttt{colortheme}
    \begin{minted}[autogobble]{latex}
        \usecolortheme{colortheme}
    \end{minted}

    \begin{block}{Color theme}
    \vspace{1em}
    \begin{tabular}{lllll}
        \texttt{default} & \cellcolor{DeepSkyBlue1}\texttt{beetle} &
        \cellcolor{DeepSkyBlue1}\texttt{dove}     & \cellcolor{magenta}\texttt{orchid} &
        \cellcolor{orange}\texttt{seahorse}\\
        \cellcolor{DeepSkyBlue1}\texttt{albatross}  &
        \cellcolor{DeepSkyBlue1}\texttt{crane}  &
        \cellcolor{DeepSkyBlue1}\texttt{fly} & \cellcolor{magenta}\texttt{rose}
        & \cellcolor{orange}\texttt{whale}\\
        \cellcolor{DeepSkyBlue1}\texttt{beaver}  &
        \cellcolor{orange}\texttt{dolphin}  & \cellcolor{magenta}\texttt{lily} & \cellcolor{DeepSkyBlue1}\texttt{seagull} & \cellcolor{DeepSkyBlue1}\texttt{wolverine}
    \end{tabular}

    \vspace{1em}
    Color themes can be {\color{DeepSkyBlue4} complete}, {\color{magenta} inner} or
    {\color{orange} outer}
    \end{block}
\end{frame}


% ------------------------------------------------------------------------------
\begin{frame}[fragile]{Outer theme}
    Select an \texttt{outertheme}

    \begin{minted}[autogobble]{latex}
        \useoutertheme{outertheme}
    \end{minted}

    The outertheme defines the head and footline of each slide

    \begin{block}{Outer themes}
    \vspace{1em}
    \begin{tabular}{llll}
    \texttt{infolines} & \texttt{miniframes} & \texttt{shadow} & \texttt{split}\\
    \texttt{sidebar} & \texttt{smoothbars} & \texttt{smoothtree} & \texttt{tree}
    \end{tabular}
    \end{block}
\end{frame}


% ------------------------------------------------------------------------------
\begin{frame}[fragile]{Inner theme}
    Select an \texttt{innertheme}

    \begin{minted}[autogobble]{latex}
        \useinnertheme{innertheme}
    \end{minted}

    The \texttt{innertheme} defines the appearance of \texttt{blocks},
    \texttt{enumerations} and other environments \emph{inside} the frame

    \begin{block}{Inner themes}
    \vspace{.5em}
    \begin{tabular}{llll}
    \texttt{rectangles} & \texttt{circles} & \texttt{inmargin} & \texttt{rounded}
    \end{tabular}
    \end{block}
\end{frame}

% ------------------------------------------------------------------------------
\begin{frame}[fragile]{Custom theme example}

The Madrid theme can be defined as

\begin{minted}[autogobble]{latex}
    \usecolortheme{whale}  % outer color
    \usecolortheme{orchid} % inner color
    \useinnertheme[shadow]{rounded}
    \useoutertheme{infolines}
    \usefonttheme{default}
\end{minted}
\end{frame}


% ------------------------------------------------------------------------------
\begin{frame}{The beamer user guide}
Everything you need to know

\vspace{.5em}
\url{http://www.math.uni.lodz.pl/~zofiawal/tex/beameruserguide.pdf}
\end{frame}


% ------------------------------------------------------------------------------
\begin{frame}<0>{}
% TODO: show
% - how to comment out blocks using \iffalse ... \fi
% - how to hide slides using <0>
% - how to hide them in the handout version only <1| handout:0>
% - how to resize stuff using resizebox/scalebox
% - how to customize individual parts (like block background,...)
% - how to create macro commands
% - adding notes
\end{frame}


% ------------------------------------------------------------------------------
\begin{frame}<0>{}
\end{frame}


% ------------------------------------------------------------------------------
\begin{frame}<0>{}
\end{frame}

\end{document}
