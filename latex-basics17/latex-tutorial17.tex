\documentclass[table]{beamer}
\usepackage[utf8]{inputenc}
\usepackage[german]{babel}
\usepackage{url}
\usepackage{minted}
\usepackage[T1]{fontenc}
\usepackage{booktabs}
\usepackage{tikz}
\usetikzlibrary{arrows,positioning,calc}

%\usepackage[table]{xcolor}
%\usetheme{Rochester}
%\usetheme{Frankfurt}
%\usecolortheme{default}
%\useoutertheme[subsection=false]{smoothbars}
\definecolor{bgg}{rgb}{0.95,0.95,0.95}
%\colorlet{bgg}{Hell-Blau!50!white}
\colorlet{bggg}{gray!30}
%\definecolor{bggg}{rgb}{0.90,0.90,0.90}
\setbeamercolor{block body}{bg=bgg,fg=black}
\setbeamercolor{block title}{bg=bggg,fg=black}

\newcommand{\latex}{\LaTeX\ }


\title{\LaTeX\ From The Ground Up}
\author{Tim Schulte}
\institute{Albert-Ludwigs-Universität Freiburg\\Grundlagen der Künstlichen Intelligenz}
\date{9. November 2017}

\begin{document}

%-------------------------------------------------------------------------------
\begin{frame}[plain]
    \titlepage
\end{frame}

%-------------------------------------------------------------------------------
\begin{frame}{Table Of Contents}
    \tableofcontents
\end{frame}

%-------------------------------------------------------------------------------
\begin{frame}<presentation:0>{What's \LaTeX ?}
    \begin{itemize}
        \item Document preparation system and markup language
        \item Build on top of the \TeX\ typsetting system % (developed by D. Knuth in 1978)
        % it is a TeX macro package allowing authors to use TeX easily
        % uses markup commands and type setting program TeX
        \item Used to create professional-looking documents % with minimal effort
        % used for publishing reports, books, scientific documents, etc.
        %\item Imposes a fundamentally different workflow (compared to WYSIWYG
        %systems)
        \item Pronounced ``LAY-tek'' or ``LAH-tek''
    \end{itemize}
\end{frame}

%-------------------------------------------------------------------------------
\begin{frame}{Why \LaTeX ?}
    \begin{center}
        \textnormal{\latex is a document preparation system and markup language.}
    \end{center}
    \begin{itemize}
        %\item creates professional-looking documents
        %\item \latex saves formatting time and effort
        \item \latex takes care of artistic details and layout design
        %\item Consistent layout, fonts, tables, etc.
        \item Mathematical formulas are exceptionally well supported
        \item Complex structures can be generated easily (footnotes, table
        of contents, list of figures, bibliography, references, etc.)
        \item Encourages writing well-structured texts, because \latex itself
        works by specifying structure
        \item Highly portable and free
    \end{itemize}
\end{frame}

%-------------------------------------------------------------------------------
\begin{frame}<presentation:0>[fragile]
    \begin{center}
    \includegraphics[width=.5\linewidth]{resource/latex-comic}
    \end{center}
\end{frame}

%-------------------------------------------------------------------------------
\begin{frame}<presentation:0>{The Philosophy Behind \latex}
    Author
    \begin{itemize}
        \item Composition and logical structuring of text
    \end{itemize}

    Typesetter
    \begin{itemize}
        \item Choice of font family, should section headings be in bold face or
        small capitals? Should they be flush left or centered? Should the text
        be justified or not? Should the notes appear at the foot of the page or
        at the end? Should the text be set in one column or two? etc. etc.
    \end{itemize}
\vspace{1em}
    \textbf{$\Rightarrow$ Focus on content instead of appearance}
\end{frame}

%-------------------------------------------------------------------------------
\begin{frame}{How \latex Works}
    \begin{center}
        \begin{tikzpicture}
\tikzstyle{n} = [thick,draw=gray!30,fill=bgg,align=center,rectangle, minimum width=7em]

\node[n] (A) at (0,0) {document text\\ and commands\\ (\emph{.tex})};
\node[n] (B) at (0,-3) {formatted\\ document\\ (\emph{.dvi, .pdf, .ps})};
\node[n] (C) at (0,-5.5) {paper\\ document};

\draw[->, thick] (A) -- node[midway]{compile} (B);
\draw[->, thick] (B) -- node[midway]{print} (C);
\draw[->, thick] (B.east) -- +(1,0)-| node[right]{edit} ($(A.east) + (1,0)$) -- (A.east);


\node[n, minimum height=12em, anchor=north] (AA) at ($(A.north)+(-6,0)$) {WYSIWYG\\ formatted\\ document};
\node[n] (CC) at (-6, -5.5) {paper\\ document};

\draw[->, thick] (AA) -- node[midway]{print} (CC);
\draw[->, thick] ($(AA.east)+(0,-1.5)$) -- ($(AA.east) + (1,-1.5)$)-|
node[right]{edit} ($(AA.east) + (1,1.5)$) -- ($(AA.east)+(0,1.5)$);
\end{tikzpicture}

    \end{center}
    %In typical "WYSIWYG" text processors, such as Microsoft Word or Word Perfect:
    % - Formatting commands are invisible.
    % - The file shows pretty much the final result.
    %LaTeX, on the other hand, is a formatter rather than a text processor:
    % - The file includes commands that define structure.
    % - The formatting commands are visible.
    % - The process requires a compiler to format the final result.
\end{frame}


%-------------------------------------------------------------------------------
\begin{frame}<presentation: 0>{Workflow}
    %First, create a (plain text) file containing regular text and \latex commands
    %to define the general structure of the document.
    % \item when writing the user uses plain text instead of formatted text
    % \item the writer uses markup tagging conventions to define the general structure of the document

    \begin{enumerate}
        \item Create a (plain text) file
        \item Edit the file's content % adding regular text and \latex commands to define the general structure of the document
        \item Compile it to create the document % (as DVI, PDF, PostScript, etc.)
        \item Display the document % (e.g. using a PDF viewer)
        \item Repeat 2. - 4.
    \end{enumerate}
\end{frame}


%-------------------------------------------------------------------------------
\begin{frame}{Setting Up \latex}
    \begin{block}{Requirements}
    \begin{itemize}
        \item \TeX\ distribution (MikTeX, TeXLive, MacTeX, \dots)
        \item Editor (TeXStudio, TeXMaker, any plain text editor)
    \end{itemize}
    \end{block}

    \begin{block}{Examples}
    \begin{itemize}
    \item \textbf{Windows}: TeXLive + TeXStudio
    \item \textbf{Linux}: TeXLive + Editor (e.g. Gedit, Vim, \dots)
    \item \textbf{Mac}: MacTeX
    \item \textbf{Browser}: ShareLaTeX
    \end{itemize}
    \end{block}
\end{frame}


%-------------------------------------------------------------------------------
\begin{frame}[fragile]{The First Document}
    \begin{block}{first.tex}
        \inputminted{latex}{resource/first.tex}
    \end{block}

    \begin{enumerate}
        \item Create a new folder and inside it a new text file called \emph{first.tex}
        \item Add the above lines to the file using a plain text editor
        \item Compile the document using \emph{pdflatex}
        \item View document using a PDF viewer
    \end{enumerate}
\end{frame}


%-------------------------------------------------------------------------------
\begin{frame}[fragile]{The First Document}
    \begin{center}
        \fbox{\includegraphics[page=1,width=.45\paperwidth]{resource/first}}
    \end{center}
\end{frame}

%------------------------------------------------------------------------------
% THE BASICS
%------------------------------------------------------------------------------
\section{Basics}

%-------------------------------------------------------------------------------
\begin{frame}[fragile]{General Document Structure}
    \inputminted[bgcolor=bgg]{latex}{resource/first-ann.tex}
    \vspace{2em}
    \begin{columns}
        \column{.5\linewidth}
        \textbf{Preamble}
        \begin{itemize}
            \item commands affecting the whole document
            \item document class, language, ...
            \item used packages
        \end{itemize}
        \column{.4\linewidth}
        \textbf{Document body}
        \begin{itemize}
            \item document content
            \item logical structure
            \item formatting instructions
        \end{itemize}
    \end{columns}

\end{frame}


%-------------------------------------------------------------------------------
\begin{frame}[fragile]{Spaces}
    \begin{columns}
        \column{.5\linewidth}
        \begin{minted}[autogobble,bgcolor=bgg]{latex}
            It does not matter
            whether you enter one
            or several     spaces
            after
            a    word.

            An empty line starts
            a new paragraph
        \end{minted}
        \column{.5\linewidth}
            It does not matter
            whether you enter one
            or several     spaces
            after
            a    word.

            An empty line starts
            a new paragraph
    \end{columns}
\end{frame}


%-------------------------------------------------------------------------------
\begin{frame}[fragile]{Special Characters}
    In \latex some characters are reserved\\
    %The following symbols are reserved characters that either have a special meaning
    %under \latex or are not available in all the fonts. If you enter them directly
    %in your text, they will normally not print, but rather coerce \latex to do
    %things you did not intend.\\
    \\
    \verb|# % ^ & _ { } ~ $ \ |\\
    \\
    Insert a backslash before these characters to get the desired result\\
    \\
    \verb|\# \% \^{} \& \_ \{ \} \~{} \$ \textbackslash|\\
    \\
    Output: \# \% \^{} \& \_ \{ \} \~{} \$ \textbackslash
\end{frame}


%-------------------------------------------------------------------------------
\begin{frame}[fragile]{Commands}
    \latex commands are of the form

    \begin{center}
        \mintinline{latex}{\commandname[opt1,opt2,...]{arg1}{arg2}...}
    \end{center}
    Some commands need an arguments given in curly braces \mintinline{latex}{{ }}

    \vspace{.5em}
    while optional parameters are given in square brackets \mintinline{latex}{[ ]}.

    \vspace{.5em}
    Examples:

    \vspace{.5em}
    \begin{minted}[bgcolor=white,autogobble]{latex}
        \documentclass{article}, {\huge profit}
    \end{minted}
\end{frame}

%-------------------------------------------------------------------------------
\begin{frame}[fragile]{Environments}
    Environments specify areas in the document where certain typesetting rules
    apply.

    \begin{center}
        \begin{minted}[autogobble,bgcolor=bgg]{latex}
        \begin{environment}
        ...
        \end{environment}
        \end{minted}
    \end{center}
\end{frame}

%-------------------------------------------------------------------------------
\begin{frame}[fragile]{Comments}
    \begin{columns}
        \column{.5\linewidth}
        \begin{minted}[autogobble,bgcolor=bgg]{latex}
        This is an % stupid
        % Better: instructive <--
        ex%
            amp%
            le.
        \end{minted}
        \column{.5\linewidth}
        This is an % stupid
        % Better: instructive <----
        ex%
        amp%
        le.
    \end{columns}
\end{frame}

%-------------------------------------------------------------------------------
\begin{frame}<presentation:0>[fragile]{Arguments}
    \begin{block}{Required argument}
        \begin{itemize}
            \item Contained in curly braces
            \item Must be included
        \end{itemize}
        Ex.: \mintinline{latex}{\documentclass{article}}
    \end{block}

    \begin{block}{Optional argument}
        \begin{itemize}
            \item Contained in square brackets
            \item Can be left out
            \item Gives you more control over the commands
        \end{itemize}
        Ex.: \mintinline{latex}{\documentclass[12pt]{article}}
    \end{block}
\end{frame}

%------------------------------------------------------------------------------
% Packages and Document Class
%------------------------------------------------------------------------------

%-------------------------------------------------------------------------------
\begin{frame}[fragile]{Document Class}
    The \emph{document class} specifies the overall layout of your document
    \begin{center}
    \mintinline{latex}{\documentclass[options]{class}}
    \end{center}
    Use \verb|article| for your paper, \verb|beamer| for your presentation.

    \begin{block}{Classes}
        \begin{tabular}{lp{20em}}
        \texttt{article} & for articles in scientific journals, presentations, short reports, program documentation\\
        \texttt{proc}    & a class for proceedings based on the article class\\
        \texttt{report}  & for longer reports containing several chapters, small books, Master's and PhD theses\\
        \texttt{book}    & for real books \\
        \texttt{letter}  & letters \\
        \texttt{beamer}  & for presentations \\
        \end{tabular}
    \end{block}
\end{frame}

%-------------------------------------------------------------------------------
\begin{frame}{Document Class Options}
    \begin{block}{Options}
    \begin{tabular}{p{10em}p{16em}}
        \texttt{10pt, 11pt, 12pt}        & Size of the main font in the document\\
        \texttt{a4paper, a5paper, letterpaper, \dots}    & The paper size\\
        \texttt{titlepage, notitlepage}  & Start a new page after the titlepage?\\
        \texttt{onecolumn, twocolumn}    & Typeset the documents in one ...
                                            \newline ... or two columns\\
        \texttt{landscape}  & Change layout to landscape mode \\
        \texttt{openright, openany}  & \dots\\
        \texttt{draft, final}  & \dots\\
    \end{tabular}
    \end{block}
\end{frame}


%-------------------------------------------------------------------------------
\begin{frame}[fragile]{Packages}
    Packages allow further customization. They are included in the preamble of
    your document
    \begin{center}
        \mintinline{latex}{\usepackage[options]{package}}
    \end{center}
    Most packages you'll need are already included in the template.

    \begin{block}{Useful packages}
        graphicx, epsfig, geometry, tikz, fancyhdr, setspace, amsmath, listings,
        xcolor, url, inputenc, babel, \dots
    \end{block}

\end{frame}

%-------------------------------------------------------------------------------
\begin{frame}[fragile]{Language Specific Packages}
    German documents require \emph{umlauts} (äöü)

    \begin{center}
    \mintinline{latex}{\usepackage[utf8]{inputenc}}
    \end{center}
    % TODO: Example (encoding - before & after)

    correct hyphenation

    \begin{center}
    \mintinline{latex}{\usepackage[german]{babel}}
    \end{center}

    and (sometimes) special fonts

    \begin{center}
    \mintinline{latex}{\usepackage[T1]{fontenc}}
    \end{center}


    Other languages are supported likewise.
    % TODO: Example (hyphenation - before & after)
    % \parbox{0pt}{\hspace{0pt}Andere übertragen das am besten den Geschwistern.}
    % \MakeUppercase{ß}
\end{frame}

%------------------------------------------------------------------------------
% FORMATTING
%------------------------------------------------------------------------------
\section{Text formatting}


%-------------------------------------------------------------------------------
\begin{frame}[fragile]{Font shapes}
    %By default \latex uses the \texttt{roman} font family

    \latex provides commands to change the
    \begin{itemize}
        \item font family

            \mintinline{latex}{\texttt} (\texttt{typewriter})
            \mintinline{latex}{\textrm} (\textrm{roman})\\
            \mintinline{latex}{\textsf} (\textsf{sans serif})

        \item font series

            \mintinline{latex}{\textbf} (\textbf{boldface})
            \mintinline{latex}{\textmd} (\textmd{medium})

        \item font shape

            \mintinline{latex}{\textup} (\textup{upright})
            \mintinline{latex}{\textit} (\textit{italic})
            \mintinline{latex}{\textsl} (\textsl{slanted})
            \mintinline{latex}{\textsc} ({\rmfamily\scshape small caps})
    \end{itemize}


    You can highlight text using \mintinline{latex}{\emph}.
\end{frame}

%-------------------------------------------------------------------------------
\begin{frame}[fragile]{Sizing text}
    \latex provides commands to change the font size

    \begin{table}
        \begin{flushleft}
        \begin{tabular}{ll}
        \mintinline{latex}{\tiny} & {\tiny tiny font}\\
        \mintinline{latex}{\scriptsize} &{\scriptsize very very small font}\\
        \mintinline{latex}{\footnotesize} &{\footnotesize very small font}\\
        \mintinline{latex}{\small} &{\small small font}\\
        \mintinline{latex}{\normalsize} &{\normalsize normal font}\\
        \mintinline{latex}{\large} &{\large slightly larger font}\\
        \mintinline{latex}{\Large} &{\Large very large font}\\
        \mintinline{latex}{\LARGE} &{\LARGE even larger font}\\
        \mintinline{latex}{\huge} &{\huge huge font}\\
        \mintinline{latex}{\Huge} &{\Huge very huge font}\\
        \end{tabular}
        \end{flushleft}
    \end{table}

    Ex.: \mintinline{latex}{\tiny tiny tiny ... \normalsize normal ...}\\
    Ex.: \mintinline{latex}{{\LARGE large large ...} normal ...}
\end{frame}


%-------------------------------------------------------------------------------
\begin{frame}[fragile]{Line- and page-breaks}
    %\begin{block}{Margins}
    %The default: between 1.5 inches and 1.875 inches
    %Set margins: \mintinline{latex}{\usepackage[margin=0.5in]{geometry}}
    %\end{block}
    \begin{itemize}
        \item Paragraphs are separated by a full blank line

        \item \mintinline{latex}{\\} ends a line explicitly without ending the whole paragraph

        \item \mintinline{latex}{\newpage} (or \mintinline{latex}{\clearpage})
            ends a page explicitly
    \end{itemize}

    %\begin{block}{Breaks}
    %Paragraphs are separated by a blank line.\\
    %Force a new line using \mintinline{latex}{\\}\\
    %Force a new page using \mintinline{latex}{\newpage} or \mintinline{latex}{\clearpage}
    %\end{block}

    %\begin{block}{Other}
    %Force a space using \mintinline{latex}{~}\\
    %Add space using \mintinline{latex}{\hspace{1in}} or \mintinline{latex}{\vspace{1in}}\\
    %Fill space using \mintinline{latex}{\hfill} or \mintinline{latex}{\vfill}
    %\end{block}
\end{frame}


%-------------------------------------------------------------------------------
\begin{frame}[fragile]{Alignment}
    Center text using the \texttt{center} environment.
    \vspace{0.5em}

    Left/right-align text using the \texttt{flushleft}/\texttt{flushright}
    environment.

    \begin{center}
    \begin{minted}[autogobble,bgcolor=bgg]{latex}
        \begin{center|flushright|flushleft}
        ...
        \end{center|flushright|flushleft}
    \end{minted}
    \end{center}

    %Use the \texttt{quote} environment to left and right indent text

    %\begin{center}
    %\begin{minted}[autogobble,bgcolor=bgg]{latex}
    %    \begin{quote} ...  \end{quote}
    %\end{minted}
    %\end{center}
\end{frame}


%------------------------------------------------------------------------------
% DOCUMENT STRUCTURE
%------------------------------------------------------------------------------
\section{Structuring}

\subsection{Title page}
%-------------------------------------------------------------------------------
\begin{frame}[fragile]{Title Page}
    \begin{block}{titlepage.tex}
    \inputminted{latex}{resource/first-titlepage.tex}
    \end{block}
\end{frame}

\subsection{Abstract}
%-------------------------------------------------------------------------------
\begin{frame}[fragile]{Abstract}
    Most research papers have an abstract.

    \begin{block}{abstract.tex}
    \begin{minted}[autogobble]{latex}
        \documentclass{article}

        \begin{document}
        \begin{abstract}
        Your abstract goes here...
        \end{abstract}
        ...
        \end{document}
    \end{minted}
    \end{block}
    The \verb|abstract| evironment is available for \verb|article|s and
    \verb|report|s.
\end{frame}


\subsection{Sectioning Commands}
%-------------------------------------------------------------------------------
\begin{frame}[fragile]{Sectioning Commands}
    There are 7 levels of depth for defining sections.

    \begin{block}{Sectioning commands and levels}
    \begin{tabular}{lcl}
        \mintinline{latex}{\part} & -1 & not in letters\\
        \mintinline{latex}{\chapter} & 0 & books and reports only\\
        \mintinline{latex}{\section} & 1 & not in letters\\
        \mintinline{latex}{\subsection} & 2 & not in letters\\
        \mintinline{latex}{\subsubsection} & 3 & not in letters\\
        \mintinline{latex}{\paragraph} & 4 & not in letters\\
        \mintinline{latex}{\subparagraph} & 5 & not in letters\\
    \end{tabular}
    \end{block}
\end{frame}

%-------------------------------------------------------------------------------
\begin{frame}[fragile]{Sectioning Commands}
\begin{columns}
\column{.6\textwidth}
        \begin{block}{sectioning.tex}
            \inputminted[bgcolor=bgg]{latex}{resource/first-sectioning.tex}
        \end{block}
\column{.4\textwidth}
        \includegraphics[page=1,trim=130 540 100 200,width=.7\paperwidth]{resource/first-sectioning}
\end{columns}
\end{frame}


%-------------------------------------------------------------------------------
\begin{frame}[fragile]{Referencing sections}
    Add labels to sections to reference them in the text.

    \begin{center}
        \begin{minted}[autogobble,bgcolor=bgg]{latex}
            \section{Results}\label{res}
            ...
            As seen in Section \ref{res} ...
        \end{minted}
    \end{center}

    \latex keeps track of section numbers for you.
\end{frame}


\subsection{Table of contents}
%-------------------------------------------------------------------------------
\begin{frame}[fragile]{Table Of Contents}
    A table of contents can be generated with
    \begin{center}
        \mintinline{latex}{\tableofcontents}
    \end{center}

    Titles of sections are added automatically to the table of
    contents\vspace{.5em}.
    You can modify the text displayed in the ToC
    \begin{center}
    \mintinline{latex}{Ex.: \section[Introduction]{Rapid Introduction To...}}
    \end{center}

    \begin{block}{Note}
        ToC entries are recorded when the document is processed.
        They are reproduced the next time the document is processed.\\
        $\Rightarrow$ \textbf{Run \emph{pdflatex} twice} to ensure that all ToC
        pagenumber references are correctly calculated.
    \end{block}
\end{frame}


\subsection{Modular documents}
%-------------------------------------------------------------------------------
\begin{frame}[fragile]{Modular documents}
When writing a book, it makes sense to split the document into multiple
\texttt{.tex} files.

\vspace{.5em}
Getting \latex to process multiple files is easy. Just use

\begin{center}
\mintinline{latex}{\input{filename}}
\end{center}

somewhere in your document, to put the contents from \texttt{filename.tex} in
place.

\vspace{.5em}
Done.

\end{frame}

%-------------------------------------------------------------------------------
\begin{frame}<presentation:0>{Bibliography}
    \begin{enumerate}
        \item Create a Bib\TeX\ file (the database)
        %\setcounter{enumi}{1}
        \item Include the \texttt{biblatex} package in the preamble
            and add the database as a \texttt{bibresource}
        \item Print the bibliography somewhere in the document body
    \end{enumerate}
    %Now you can cite sources from the database throughout your document text
\end{frame}


\subsection{Citations, Bib\TeX}
%-------------------------------------------------------------------------------
\begin{frame}[fragile]{Bib\TeX\ file format}
    \begin{block}{mybib.bib}
    \begin{minted}[autogobble]{latex}

        @article{maxmustermann,
            author = {Mustermann, Max},
            title = {Mustermann on topics of interest},
            journal = {Journal of Mustermann},
            volume = 46,
            year = 1993,
            number = 2,
            pages = {35--53}
        }
        ...
    \end{minted}
    \end{block}

    For an indepth description of the database format see\\
    \scriptsize
    \url{https://en.wikibooks.org/wiki/LaTeX/Bibliography_Management#BibTeX}
\end{frame}


%-------------------------------------------------------------------------------
\begin{frame}[fragile]{The Bib\LaTeX\ package}
    \begin{center}
        \begin{minted}[autogobble,bgcolor=bgg]{latex}
            % in the preamble
            \usepackage[bibencoding=utf8,
                backend=biber, style=numeric]{biblatex}
            \addbibresource{mybib.bib} % or

            % where the bibliography will be printed
            \printbibliography
        \end{minted}
    \end{center}

    There are other styles, like \verb+alphabetic+, \verb+authoryear+, \dots
    % \bibliography{mybib} [deprecated]
    % \nocite{*}
\end{frame}

%-------------------------------------------------------------------------------
\begin{frame}[fragile]{Citations}
    Add references to your document with
    \begin{center}
        \mintinline{latex}{\cite}
    \end{center}

    \begin{block}{Example}
        \mintinline{latex}{Redundancy \cite{maxmustermann}}\\
        \mintinline{latex}{...methodology \cite{entry1, entry2, ...}}
    \end{block}
\end{frame}

%-------------------------------------------------------------------------------
\begin{frame}[fragile]{Bibliography summary}
    \begin{enumerate}
        \item Create a Bib\TeX\ file (the database)
        %\setcounter{enumi}{1}
        \item Include the \texttt{biblatex} package in the preamble
            and add the database as a \texttt{bibresource}
        \item Print the bibliography somewhere in the document body
    \end{enumerate}

    \begin{block}{Note}
        To build the bibliography, first compile the document, then
        generate the necessary \texttt{.bbl} file, and compile the document
        again.
        \begin{minted}[autogobble]{bash}
            > pdflatex <myfile.tex>
            > biber <myfile>
            > pdflatex <myfile.tex>
        \end{minted}
        If you use an IDE it will probably take care of this for you.
    \end{block}
\end{frame}

%-------------------------------------------------------------------------------
\iffalse
\begin{frame}[fragile]{Referencing}
    \begin{block}{Footnotes}
    \mintinline{latex}{...telephony\footnote{Phony telephones}}
    \end{block}
\end{frame}
\fi

%------------------------------------------------------------------------------
% ENVIRONMENTS
%------------------------------------------------------------------------------
\section{Environments}

\subsection{Lists}
%-------------------------------------------------------------------------------
\begin{frame}[fragile]{Lists}
    There are two basic types of lists (which can also be nested).

    \begin{block}{List environments}
    \begin{columns}
    \column{.4\linewidth}
        \begin{minted}[autogobble]{latex}
            \begin{itemize}
                \item This list
                \item is
                \item \emph{unordered}
            \end{itemize}
        \end{minted}
    \column{.3\linewidth}
            \begin{itemize}
                \item This list
                \item is
                \item \emph{unordered}
            \end{itemize}
    \end{columns}
    %\end{block}

    %\begin{block}{Ordered List}
    \begin{columns}
    \column{.4\linewidth}
        \begin{minted}[autogobble]{latex}
            \begin{enumerate}
                \item This list
                \item is
                \item \emph{ordered}
            \end{enumerate}
        \end{minted}
    \column{.3\linewidth}
            \begin{enumerate}
                \item This list
                \item is
                \item \emph{ordered}
            \end{enumerate}
    \end{columns}
    \end{block}
\end{frame}


\subsection{Figures}
%-------------------------------------------------------------------------------
\begin{frame}[fragile]{Figures}
    Use the \verb|graphicx| package + %in the preamble
    \verb|figure| environment to embed pictures.
    \begin{center}
        \mintinline{latex}{\usepackage{graphicx}}
    \end{center}

    \begin{block}{Figure environment}
            \begin{minted}[autogobble,bgcolor=bgg]{latex}
                \begin{figure}
                    %\centering
                    \includegraphics[width=.35\linewidth]{gopher}
                    \caption{A gopher.}
                    \label{fig:gopher1}
                \end{figure}
            \end{minted}
    \end{block}
\end{frame}

%-------------------------------------------------------------------------------
\iffalse
\begin{frame}[fragile]{Figures}{Example}
    \inputminted[bgcolor=bgg]{latex}{resource/figures.tex}
\end{frame}
\fi

%-------------------------------------------------------------------------------
\begin{frame}[fragile]{Figure Positioning}
    %\begin{itemize}
        %\item
    \latex may choose to put the picture on a different location\vspace{.5em}.
            % i.e. the next page, or any other page where it finds sufficient space.
        %\item

            Adding [h!] behind the figure environment forces the figure to be
            shown at the exact location in the document.
            \begin{center}
            \mintinline{latex}{\begin{figure}[h!]}
            \end{center}
    %\end{itemize}


    \begin{block}{Positioning flags}
        \begin{tabular}{ll}
        \verb|h| (here) & same location\\
        \verb|t| (top) & top of page\\
        \verb|b| (bottom) & bottom of page\\
        \verb|p| (page) & on an extra page\\
        \verb|!| (override) & will force the specified location
        \end{tabular}
    \end{block}
\end{frame}

%-------------------------------------------------------------------------------
\begin{frame}{Tables}
    Demo\\
    \\
    See
    \url{https://en.wikibooks.org/wiki/LaTeX/Tables}
\end{frame}


\subsection{Verbatim}
%-------------------------------------------------------------------------------
\begin{frame}[fragile]{Verbatim}
To introduce text that won't be interpreted by the compiler, use the
\texttt{verbatim} environment.\\

\begin{columns}
    \column{.5\linewidth}
    \begin{minted}[autogobble,bgcolor=bgg]{latex}
    \begin{verbatim}
    The verbatim environment
      simply reproduces every
     character you input,
    including all  s p a c e s!
    % & { }
    \end{verbatim}
    \end{minted}
    \column{.45\linewidth}
\begin{verbatim}
The verbatim environment
  simply reproduces every
 character you input,
including all  s p a c e s!
% & { }
\end{verbatim}
\end{columns}
\end{frame}

%------------------------------------------------------------------------------
% MATH
%------------------------------------------------------------------------------
\subsection{Math}

%-------------------------------------------------------------------------------
\begin{frame}{Math Environments}
    \latex needs to know when text is mathematical.

    \vspace{.5em}
    There are two main environments

    \vspace{1em}
    \begin{enumerate}
        \item Inline formulas (within text)\\
        \mintinline{latex}{\(...\)}\\

    \item Displayed equations (separated from text)\\
        \mintinline{latex}{\[...\]}\\
    \end{enumerate}
\end{frame}

%-------------------------------------------------------------------------------
\begin{frame}[fragile]{Symbols}
    Mathematics has symbols. Some can be accessed directly

    \begin{center}
    \mintinline{latex}{+ - = ! / ( ) [ ] < > | ' :}
    \end{center}

    \vspace{.5em}
    Others require distinct commands. For instance

    \begin{center}
    \begin{minted}[autogobble,bgcolor=bgg]{latex}
    \( \forall x \in X, \quad \exists y \leq \epsilon \)
    \end{minted}
    \end{center}

    Produces: \(\forall x \in X, \quad \exists y \leq \epsilon\)

    \vspace{1em}
    For more advanced mathematic operators see\\
    \url{https://en.wikibooks.org/wiki/LaTeX/Mathematics}
    \url{http://www.hostmath.com}
\end{frame}

%------------------------------------------------------------------------------
% COMMON MISTAKES
%------------------------------------------------------------------------------
\section{Common Mistakes}

%-------------------------------------------------------------------------------
\begin{frame}[fragile]{Common Mistakes}
    A space right after a period following a lowercase letter by default ends a
    sentence and LaTeX inserts an extra whitespace. There are several occasions
    where you do not want to have the default behaviour.

    \begin{block}{Example}
        \begin{minted}[autogobble]{latex}
    Ms. Bean is \ldots\\
    Ms.\ Bean is \ldots

    I left at 12:00 P.M. In \ldots\\
    I left at 12:00 P.M\@. In \ldots

    \LaTeX is fun.\\
    \LaTeX\ is fun.
        \end{minted}
    \end{block}
\end{frame}

%-------------------------------------------------------------------------------
\begin{frame}[fragile]{Common Mistakes}
    In math mode variables with more than two characters must be wrapped inside
    \mintinline{latex}{\mbox or \mathit} environments.\\
    \begin{block}{Example}
        \begin{minted}[autogobble]{latex}
            \[
            gbfs = 35\\
            \mathit{gbfs} = 35
            \]
        \end{minted}
    \end{block}
\end{frame}

\begin{frame}[fragile]{Common Mistakes}
    Using the wrong quotation marks.

    \vspace{1em}
    \begin{columns}
        \column{.4\linewidth}
        \begin{minted}[autogobble,bgcolor=bgg]{latex}
        `American'
        ``American''
        ,,German``
        <<French>>
        \end{minted}
        \column{.4\linewidth}
        `American'\\
        ``American''\\
        ,,German``\\
        <<French>>
    \end{columns}

    \vspace{1em}
    For european quoting style use T1 font encoding:

    \mintinline{latex}{\usepackage[T1]{fontenc}}
\end{frame}

%-------------------------------------------------------------------------------
\begin{frame}[fragile]{Common Mistakes}
    Not using UTF8

    \vspace{0.5em}
    Just put
    \begin{center}
    \mintinline{latex}{\usepackage[utf8]{inputenc}}
    \end{center}
    in your preamble and you're done.
    %  for magic to write foreign characters!
\end{frame}

%-------------------------------------------------------------------------------
\begin{frame}[fragile]{Common Mistakes}
    Not using version management software (git, mercurial, svn)
\end{frame}

\end{document}
